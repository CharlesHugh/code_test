%!TEX program = xelatex
\documentclass{article}
\usepackage{fontspec, xunicode, xltxtra,metalogo,ulem,verbatim,paralist,amsmath}
\setmainfont{Hiragino Sans GB}
\title{Title}
\author{name}
\begin{document}
\maketitle{title}
\section{Introduction}
This is where you will write your content. 在这里写上内容。
\# \$ \^ \& \_ \{ \} \~ \textbackslash \% \\
\TeX  \LaTeX\\
\emph{强调:emphasis}\\  
\uline{下划线:underline}\\  
\uwave{波浪线:waveline}\\ 
\sout{删除线:strike-out}\\

computer-aided\\ 
1840--2010\\
to be---or not to be\\ 
$1-1=0$

\begin{flushleft}
居左\\
段落 
\end{flushleft}

\begin{center}
居中\\
段落 
\end{center}

\begin{flushright}
居右\\
段落 
\end{flushright}

\footnote

\marginpar


\newpage

   \begin{comment} 
   verbatim宏包\\
   comment... 
   \end{comment}

\begin{quote}
引文两端都缩进。引文两端都缩进。引文两端都缩进。 \\
引文两端都缩进。引文两端都缩进。引文两端都缩进。\\
\end{quote}

\begin{quotation}
引文两端缩进,首行增加缩进。引文两端缩进,首行增加缩进。\\
引文两端缩进,首行增加缩进。引文两端缩进,首行增加缩进。\\
\end{quotation}

\begin{verse}
引文两端缩进,第二行起增加缩进。引文两端缩进,第二行起增加缩进。\\
引文两端缩进,第二行起增加缩进。引文两端缩进,第二行起增加缩进。 \\
\end{verse}

%列表
\begin{itemize} 
\item C++
\item Java
\item HTML 
\end{itemize}

\begin{enumerate} 
\item C++
\item Java
\item HTML 
\end{enumerate}

\begin{description}
\item[C++] 编程语言
\item[Java] 编程语言
\item[HTML] 标记语言 
\end{description}

%行距较小的paralist宏包
\begin{compactitem} 
\item C++
\item Java
\item HTML 
\end{compactitem}

\begin{compactenum} 
\item C++
\item Java
\item HTML 
\end{compactenum}

\begin{compactdesc}
\item[C++] 编程语言,
\item[Java] 编程语言,
\item[HTML] 标记语言。 
\end{compactdesc}

\begin{inparaitem} 
\item C++
\item Java
\item HTML 
\end{inparaitem}

\begin{inparaenum} 
\item C++
\item Java
\item HTML 
\end{inparaenum}

\newpage

\mbox{010 6278 5001} 
\fbox{010 6278 5001}

%语法:[宽度][对齐方式]{内容}
\makebox[100pt][c]{仪仗队}
\framebox[100pt][s]{仪仗队}

%高级盒子
\fbox{% 
  \parbox[c][36pt][t]{170pt}{
  锦瑟无端五十弦,一弦一柱思华年。\\庄生晓梦迷蝴蝶,望帝春心托杜鹃。
  }% 
}
\hfill 
\fbox{%
  \begin{minipage}[c][36pt][b]{170pt}
  沧海月明珠有泪,蓝田日暖玉生烟。\\此情可待成追忆,只是当时已惘然。
  \end{minipage}% 
  }






\end{document}
